\section*{Závěr}
\addcontentsline{toc}{section}{Závěr}
\label{sec:ending}

{\sloppy
Během 74 dnů aktivního vývoje se mi podařilo navrhnout, naprogramovat a otestovat \textbf{plně funkční herní interpret} běžící jako server sloužící ke správě her ve formě map, jimiž se snaží boti dostat ke svému cíli, pokladu - tím se mi podařilo kompletně splnit \emph{první bod zadání}.

\emph{Druhý bod zadání} jsem splnil implementací vlastní konfigurace, kdy pomocí konfiguračního formuláře v~administraci lze měnit \textbf{jakékoliv} konfigurovatelné \textbf{parametry hry} a jako bonus je doplněna možnost i \textbf{mazat neaktivní hry} a možnost si \textbf{zobrazit detail} jakékoliv hry v~aplikaci.

\emph{Třetí bod zadání} pojednává o~stylech hry. Tento bod byl splněn v~rámci druhého bodu implementací počtu botů na hru a možností hry \textbf{více botů} z~\textbf{jednoho zařízení} a zároveň možnosti hry \textbf{více botů} z~\textbf{více různých zařízení}. Klientské zařízení může tvořit cokoliv od osobního počítače, přes chytré mobilní telefony, servery až k~minipočítačům typu Raspberry Pi - záleží pouze na použité programové platformě daného klienta.

Ve \emph{čtvrtém bodu zadání} se žádá zorganizování soutěže. Kromě této práce vznikl i \textbf{tutoriál pro řešení klienta} (lze jej vidět na kořenové adrese běžícího herního serveru), který použili dva programátoři k~vytvoření vlastního klienta (jeden v~jazyku \emph{Java}, druhý v~\emph{Lua})- klientské programy jsou přiloženy na CD. K~uspořádání soutěže by bylo vhodné mít k~dispozici alespoň 4 klienty, je tedy prodloužen termín pro naprogramování a samotná formální soutěž bude uspořádána dodatečně.

Během práce jsem si doplnil znalosti programování v~jazyku Python, naučil se reálně používat framework Flask a především, naučil jsem se psát \textbf{jednotkové testy pro webové aplikace} v~Pythonu. V~době dokončení projektu bylo pokryto \textbf{2423 řádků z~celkových 2478}, což dává \textbf{výborných 97.82 \%} pokrytí kódu jednotkovými testy.

\subsection*{Potenciální budoucí vylepšení}

\begin{itemize}
 \item pro bezpečnější použití by znovu navrhnout a vylepšit systém autorizace k~přístupu k~ovládání bota - existují bezpečnější metody než ID bota použitý jako soukromý klíč přenášený po HTTP protokolu
 \item přidat herní log, ze kterého bude možné zrekonstruovat celý průběh hry od vygenerování až po vítězství jednoho z~botů  
 \item přidat bonusové herní bloky, teleporty bloků, systém min nebo obranná elektromagnetická pole
 \item reimplementovat princip konfigurací pro plně dynamickou množinu parametrů vytvářenou čiště klientem
\end{itemize}
}

\section{Teoretický úvod}

\subsection{Rozbor zadání}
\todo{scan zadání + rozbor jednotlivých bodů}
\subsection{Návrh architektury aplikace}

\subsubsection{Návrh interní struktury}

\textit{
	V této sekci se budu zabývat pouze teoretickým návrhem jednotlivých datových a řídící struktur. Popisu použitých algoritmů a jednotlivých implementačních rozhodnutí se budu věnovat v sekci \fullref{sec:implementation}.
}

O interní reprezentaci jedné instance hry se stará třída \ic|Game|, která je zodpovědná za striktní přístup klienta do mapy. Zpracovává jednotlivé herní akce klientů, kontroluje jejich validitu a aplikuje změny do mapy. \ic|Game| je sama schopna se vyexportovat do slovníku, který je následně převeden do formátu \nameref{subsec:json}, jenž je následně distribuován ke klientovi. Při herních akcích překládá výjimky vyvolané interně uloženou mapou na ty zvenku známé. V případě módu hry s tahy hlídá \ic|Game| pořadí jednotlivých botů.

Třída \ic|Game| interně využívá třídu \ic|Map| k uchování stavu hry. Třída \ic|Map| je v podstatě pouze zapouzdřený dvourozměrný kontejner reprezentující vlastní mapu. Je zodpovědná za validní přístup - tzn. mj. ošetřuje stavy pro nevalidní přístup mimo rozsah mapy.

Společným předkem pro všechny entity uložené v mapě je třída \ic|Field|. Jejím nejprimitivnějším potomkem je prázdné pole \ic|EmptyField|.

Existenci bota v mapě reprezentuje třída \ic|BotField|, která je zopovědná za udržení jeho prostorové orientace a umí se na místě otáčet. V případě, že se jedná o hru s bateriemi, je tato třída nahrazena třídou \ic|LaserBatteryBotField|, která se kromě orientace stará i o stav baterie. Ten je řízen pomocí metod \ic|LaserBatteryBotField.charge()| pro nabíjení, resp. \ic|drain()| pro vybíjení. Výjimka \ic|CriticalBatteryLevel| je vyvolána v případě, kdy by mělo dojít k vybití baterie pod nulovou úrove\v{n}.

Mezi další pomomky třídy \ic|Field| patří \ic|BlockField| reprezentující v mapě pole pevného bloku a \ic|TreasureField| zastupující poklad ve hře.

Třída, která kontroluje jednotlivé instance \ic|Game| v aplikaci se příznačně nazývá \ic|GameController|. Je zodpovědná delegování klientského požadavku na herní akci na odpovídající instanci hry. Zároveň zodpovídá za vytváření her, resp. na svém vstupu příjmá instanci potomka třídy \ic|BaseConfiguration|, kterou následně předá do singletonu třídy \ic|MapFactory|, která sestaví instanci mapy.

\ic|MapFactory| je třída zodpovědná za vygenerování mapy v závislosti na předané configuraci. Plný výčet parametů konfigurace lze najít v tabulce \fullref{table:conf-parameters}.

\begin{table}[H]
	\centering
	\begin{tabular}{ l | l | l | l }
		český název & název parametru & vysvětlení & datový typ \\
		\hline
		šířka mapy & \ic|map_width| & počet bloků na šířku mapy & int \\
		výška mapy & \ic|map_heigth| & počet bloků na výšku mapy & int \\
		počet botů & \ic|bots| & maximální počet botů ve hře & int \\
		počet bloků & \ic|blocks| & maximální počet bloků ve hře & int \\
		počet pokladů & \ic|treasures| & počet pokladů ve hře & int \\
		hra s tahy & \ic|rounded_game| & hra botů v pořadí & bool \\
		hra s bateriemi & \ic|battery_game| & hra botů s bateriemi & bool \\
		hra s lasery & \ic|laser_game| & hra botů s možností laseru & bool \\
	\end{tabular}
	\caption{Seznam možných parametrů herní konfigurace s parametry}
	\label{table:conf-parameters}
\end{table}

\subsubsection{Návrh vnějšího rozhraní}
\todo{probrat URL, metody v závislosti na URL}

\subsection{Administrace}
\todo{změna konfigurace z formuláře, mazání her, náhled na hru}


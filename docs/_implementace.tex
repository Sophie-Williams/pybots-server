\section{Interní implementace aplikace}
\label{sec:implementation}

\subsection{Pomocná třída Exportable}

Abstraktní třída \ic|Exportable| je určena k identifikování všech tříd, jejichž instance je schopna se vyexportovat. Obsahuje jedinou metodu \ic|export|, která by v potomcích měla vracet jedině takové hednoty, které je možné překonvertovat do formátu \nameref{subsec:json} - v naprosté většině případů se v aplikaci k exportu používá vestavěný datový typ \ic|dict|.

\subsection{Enumerace Orientation}

Tato enumerace dědící ze třídy \ic|IntEnum| z balíčku \ic|enum| zajištuje především sjednocení udávaných orientací v aplikaci, v některých případech označuje i směr. Je označena dekorátorem \ic|unique|, který zajištuje unikátnost jednotlivých hodnot v enumeraci. Tato enumerace obsahuje klíče \ic|NORTH|, \ic|EAST|, \ic|SOUTH| a \ic|WEST| s hodnotami v rozsahu od 1 do 4. Jako jednoduché utility jsou ve třídě metody (resp. vlastnosti díky dekorátoru \ic|@property|) \ic|is_horizontal| a \ic|is_vertical| ověřující horizontální, resp. vertikální směr. Výhoda v předkovi \ic|IntEnum|, resp. \ic|Enum| tkví v možnosti testovat shodnost pomocí operátoru rovnosti a konstruovat tyto objekty pomocí číselné hodnoty, viz následující ukázka.

\begin{lstlisting}[caption={Výhody třídy Enum}]
orientation_by_key = Orientation.NORTH
orientation_by_value = Orientation(0)

assert orientation_by_value == orientation_by_key
\end{lstlisting}

\subsection{Enumerace Action}

Enumerace \ic|Action| se v aplikaci používá jako jednoznačný identifikátor akce pro jednotlivé boty. Je stejně jako \ic|Orientation| odekorována pomocí \ic|@unique|, ale narozdíl od ní se nejedná o číselnou enumeraci, ale o enumeraci řetězců - pro snažší identifikaci v rámci požadavků na server.
Mezi její hodnoty patří \ic|STEP| pro pohyb robota, \ic|TURN_LEFT| a \ic|TURN_RIGHT| pro jeho otáčení, \ic|WAIT| pro čekání na místě (a nabití baterie) a \ic|LASER_BEAM| pro aktivaci laserového paprsku.

\subsection{Enumerace Field}

\todo{Enumerace Field a její klíče}

\subsection{Enumerace ReponseState}

\todo{popsat ResponseState}

\subsection{Kontejnerová třída Map}

\ic|Map| je implementována jako dvourozměrná instance datového typu \ic|list|. Při inicializování objektu je ihned v konstruktoru (metoda \ic|__init__|) vytvořen dvourozměrný seznam - první rozměr pro výšku, druhý pro šířku. Oba tyto parametry jsou předány konstruktorem a je ověřena jejich nenulovost. Celý kontejner je naplněn instancemi třídy \ic|EmptyField|. Mezi další metody této třídy patří \ic|get_field_occurrences|, vracející seznam souřadnic, na kterých se nachází instance předáné třídy. Slouží k vyhledávání nad mapou, ale bohužel je její složitost vzhledem k implementaci $O(n^2)$. Metoda \ic|get_next_field| je určena k získávání vedlejšího pole dle zadaných souřadnic a instance třídy \ic|Orientation|. V případě nemožnosti získat další pole ve směru na kraji mapy je vráceno \ic|None|. 

Níže uvedený příklad znázorňuje přetížené indexování instancí třídy \ic|Map| pomocí pozice uložené v vestavěném datovém typu \ic|tuple| (nebo jakýkoliv jiný objekt, který implementuje metodu \ic|__iter__|). Výsledkem je instance prázdného herního pole.

\begin{lstlisting}[caption={Přetížené indexování třídy Map}]
position = 3, 2
game_map = Map(width=10, height=10)

assert isinstance(
	game_map[position],
	EmptyField
)
\end{lstlisting}

\subsection{Třídy reprezentující herní bloky}

Na tomto místě je nutno poznamenat, že potomci třídy \ic|Field|, jenž jsou umisťováni do mapy, nejsou zodpovědni za své umístění, tzv. neuchovávají žádné informace o své pozici, ale pouze informace o svém stavu, jako jsou například orientace, jméno nebo stav baterie. 

\subsubsection{Společný abstraktní předek Field}

\begin{sloppypar}
	Pro všechny instance polí existuje společný abstraktní předek - třída \ic|Field|, ze které dědí všechna možná pole umístitelná do mapy. Samotná třída je potomkem třídy \ic|Exportable|, znamená to tedy, že každý z potomků této třídy musí implementovat metodu \ic|export|, což zajištuje možnost reprezentace herní struktury i mimo aplikaci.
\end{sloppypar}

\subsubsection{Prázdné pole EmptyField}

Jedná se o prázdnou variantu herního pole, je zodpovědná pouze za svůj korektní export.

\subsubsection{Pole bloku BlockField}

Třída \ic|BlockField| je ve své jednoduchosti velmi podobná prázdnému poli. Jediný rozdíl skrývá rozdílný přístup třídy \ic|Game| při herních akcích - na pole s objektem této třídy není možné vstoupit botem, avšak při zapnutém parametru pro laser hru je možno tento blok zničit. Po zničení je instance pole s blokem nahrazena instancí prázdného pole.

\subsubsection{Pole pro poklad TreasureField}

Ve své podstatě se jedná o pole jako každé jiné v mapě, avšak má jednu spicální vlastnost. Při vstoupení na něj botem je vyvolána výjimka \ic|GameFinished| (o struktuře a systému výjimek v sekci \fullref{subsec:custom-exceptions}) a hra končí. 

\subsubsection{Pole pro boty BotField a LaserBatteryBotField}
\todo{popsat bot fieldy, především zodpovědnosti}

\subsection{Systém a struktura vlastních výjimek}
\label{subsec:custom-exceptions}

Proces zpracování herní akce je záležitost, při které může dojít až k několika desítkám nevalidních nebo hru ukončujících stavů, jako např. bot mimo herní mapu, pohyb bota, který není na řadě, požadavek na neznámého bota, požadavek na neznámou mapu, nesprávná herní akce pro kontext hry anebo hra s již zaplněnými poli pro boty. Proto je v aplikaci navržen systém výjimek, které jsou vyhazovány v momentech, kdy je již proces zpracování akce zahájen.

\begin{table}[H]
	\newcolumntype{B}{>{\collectcell\icmacro}r<{\endcollectcell}}
	\renewcommand{\arraystretch}{1.2}
	\centering
	
	\begin{tabular}{ B || p{.6\textwidth} }
		OutOfMapError & Tuto výjimku používá třída \ic|Map| k indikaci přístupu mimo rozsah mapy. Kromě utility \ic|get_positions_in_row| ji používá především třída \ic|Game| při procesu zpracování herní akce. \\

		BotNotOnTurn & Výjimka tohoto typu je určena k indikaci stavu, kdy je vznesen požadavek na tah bota, který ve hře se zapnutým parametrem \ic|rounded_game| není na řadě. \\

		NoFreeBots & Tento typ vyjímky slouží k rozpoznání stavu, kdy se řídící třída pokusí přidat bota do hry, v níž již není volné pole pro bota. \\

		ConfigurationError & \ic|ConfigurationError| se používá pro signalizaci stavu nevalidní konfigurace. Ve dvou případech je užívána třídou \ic|MapFactory|, jenž její pomocí rozlišuje stavy při nevalidním požadavku na vytvoření mapy (nemožnost umístit příliš moc herních bloků do příliš malé mapy) a nelogickou konfiguraci parametrů \ic|battery_game = True| a \ic|laser_game = False|. Pomocí této výjimky se také kontroluje validita configuračního formuláře - více tomto formuláři v sekci \fullref{subsec:administration}. \\

		GameFinished & Tato výjimka ukončuje hru - je vyvolána třídou \ic|Game| při vstoupení bota na pole třídy \ic|TreasureField|. Tuto výjimku očekává instance třídy \ic|GameController| v případě jejího zachycení zaznamenává tuto hru za ukončenou a bota, jenž provedl tento krok za výherce hry. \\

		CriticalBatteryLevel & Výjimku, jenž je potomkem výjimky \ic|MovementError|, vyvolává třída pole \ic|LaserBatteryBotField| v případě požadavku na vybití baterie pod nulovou úroveň. Vzhledem k jejímu rodiči je tedy tato výjimka odchytávána jako nemožnost provést herní akci resp. nemožnost provést krok ve hře. \\
	\end{tabular}
	\caption{Seznam vlastních výjimek a jejich popis}
\end{table}

\subsection{Testy}
\todo{popsat identickou strukturu testů, systém volání flasku}

\subsubsection{TestCase}

V testech pro tuto aplikaci používám vlastní poděděnou třídu pro množinu testovacích metod \ic|TestCase|. Je rozšířená o tyto metody s těmito hlavičkami:
\begin{description}
	\item \ic|assertIsInMap(self, game_map, field_cls, expected_count=1)| \\
	Tato metoda rozšiřuje testovací aserce v původní testovací třídě \ic|TestCase| o možnost testovat herní mapu (instanci třídy \ic|Map|) na určitý počet instací herních polí, resp. instancí třídy předané v parametru \ic|field_cls|. Ve výsledku je provedena aserce počtu nalezených těchto bloků vůči hodnotě parametru \ic|expected_count| s výchozí hodnotou \ic|1|.
	
    \item \ic|test_client(self)| \\
    \todo{dopsat text k metodám testCase}

    \item \ic|set_conf_to_test_client(self, conf, client)| \\

    \item \ic|setUp(self)| \\

    \item \ic|tearDown(self)| \\

\end{description}
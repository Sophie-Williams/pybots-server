\section{Interní implementace aplikace}
\label{sec:implementation}

\subsection{Pomocná třída $Exportable$}

Abstraktní třída \ic|Exportable| je určena k~identifikování všech tříd, jejichž instance je schopna se vyexportovat. Obsahuje jedinou metodu \ic|export|, která by v~potomcích měla vracet jedině takové hednoty, které je možné překonvertovat do formátu \nameref{subsec:json} - v~naprosté většině případů se v~aplikaci k~exportu používá vestavěný datový typ \ic|dict|.

\subsection{Enumerace $Orientation$}

Tato enumerace dědící ze třídy \ic|IntEnum| z~balíčku \ic|enum| zajištuje především sjednocení udávaných orientací v~aplikaci, v~některých případech označuje i směr. Je označena dekorátorem \ic|unique|, který zajištuje unikátnost jednotlivých hodnot v~enumeraci. Tato enumerace obsahuje klíče \ic|NORTH|, \ic|EAST|, \ic|SOUTH| a \ic|WEST| s~hodnotami v~rozsahu od 1 do 4. Jako jednoduché utility jsou ve třídě metody (resp. vlastnosti díky dekorátoru \ic|@property|) \ic|is_horizontal| a \ic|is_vertical| ověřující horizontální, resp. vertikální směr. Výhoda v~předkovi \ic|IntEnum|, resp. \ic|Enum| tkví v~možnosti testovat shodnost pomocí operátoru rovnosti a konstruovat tyto objekty pomocí číselné hodnoty, viz následující ukázka.

\begin{lstlisting}[caption={Výhody třídy $Enum$}]
orientation_by_key = Orientation.NORTH
orientation_by_value = Orientation(0)

assert orientation_by_value == orientation_by_key
\end{lstlisting}

\subsection{Enumerace $Action$}
\label{subsec:action-enum}
Enumerace \ic|Action| se v~aplikaci používá jako jednoznačný identifikátor akce pro jednotlivé boty. Je stejně jako \ic|Orientation| odekorována pomocí \ic|@unique|, ale narozdíl od ní se nejedná o~číselnou enumeraci, ale o~enumeraci řetězců - pro snažší identifikaci v~rámci požadavků na server.
Mezi její hodnoty patří \ic|STEP| pro pohyb robota, \ic|TURN_LEFT| a \ic|TURN_RIGHT| pro jeho otáčení, \ic|WAIT| pro čekání na místě (a nabití baterie) a \ic|LASER_BEAM| pro aktivaci laserového paprsku.

\subsection{Enumerace $Field$}

Jedná se o~enumeraci typově rozlišující od sebe jednotlivé potomky abstraktní třídy \ic|Field| nacházejících se v~mapě. Hodnoty této enumerace se poté nacházejí ve vyexportované mapě pro klienty. Jednotlivé hodnoty této enumerace jsou k~vidění v~tabulce\fullref{table:field-enum-values}.

\begin{table}[H]
    \caption{Seznam hodnot enumerace $Field$}
    \label{table:field-enum-values}
	\centering
	\begin{tabular}{ r r | l }
		hodnota & klíč & odpovídající třída \\
		\hline
		$0$ & \ic|EMPTY| & \ic|EmptyField| \\
		$1$ & \ic|TREASURE| & \ic|TreasureField| \\
		$2$ & \ic|BOT| & \ic|BotField| \\
		$3$ & \ic|BLOCK| & \ic|BlockField| \\
		$4$ & \ic|BATTERY_BOT| & \ic|LaserBatteryBotField| \\
	\end{tabular}
\end{table}

\subsection{Enumerace $ReponseState$}

Tato enumerace je určena pro definici návratových statusů pro klientské požadavky. Definuje textovou identifikaci stavu společně s~HTTP návratovým kódem. Kromě hodnoty $200$ značící korektní stav je použita také hodnota $404$ značící nenalezený objekt. Jednotlivé hodnoty lze nalézt v~tabulce\fullref{table:response-state-enum-values}.

\begin{table}[H]
    \caption{Seznam hodnot enumerace $ResponseState$}
    \label{table:response-state-enum-values}
	\centering
	\begin{tabular}{ r r | l }
		klíč & hodnota $state$ & návratový kód \\
		\hline
		\ic|GAME_WON| & \ic|game_won| & $200$ \\
		\ic|GAME_LOST| & \ic|game_lost| & $200$ \\
		\ic|UNKNOWN_BOT| & \ic|unknown_bot| & $404$ \\
		\ic|INVALID_ACTION| & \ic|invalid_action| & $404$ \\
		\ic|ACTION_ERROR| & \ic|action_error| & $404$ \\
		\ic|MOVEMENT_ERROR| & \ic|movement_error| & $200$ \\
		\ic|CRITICAL_BATTERY_LEVEL| & \ic|critical_battery_level| & $200$ \\
		\ic|BOT_NOT_ON_TURN| & \ic|bot_not_on_turn| & $200$ \\
		\ic|MOVEMENT_SUCCESS| & \ic|movement_success| & $200$ \\
	\end{tabular}
\end{table}

\subsection{Kontejnerová třída $Map$}

\begin{sloppypar}
    \ic|Map| je implementována jako dvourozměrná instance datového typu \ic|list|. Při inicializování objektu je ihned v~konstruktoru (metoda \ic|__init__|) vytvořen dvourozměrný seznam - první rozměr pro výšku, druhý pro šířku. Oba tyto parametry jsou předány konstruktorem a je ověřena jejich nenulovost. Celý kontejner je naplněn instancemi třídy \ic|EmptyField|. Mezi další metody této třídy patří \ic|get_field_occurrences|, vracející seznam souřadnic, na kterých se nachází instance předáné třídy. Slouží k~vyhledávání nad mapou, ale bohužel je její složitost vzhledem k~implementaci $O(n^2)$. Metoda \ic|get_next_field| je určena k~získávání vedlejšího pole dle zadaných souřadnic a instance třídy \ic|Orientation|. V~případě nemožnosti získat další pole ve směru na kraji mapy je vráceno \ic|None|.
\end{sloppypar}

Zdrojový kód\fullref{lst:map-indexing} znázorňuje přetížené indexování instancí třídy \ic|Map| pomocí pozice uložené v~vestavěném datovém typu \ic|tuple| (nebo jakýkoliv jiný objekt, který implementuje metodu \ic|__iter__|). Výsledkem je instance prázdného herního pole.

\begin{lstlisting}[caption={Přetížené indexování třídy $Map$},label={lst:map-indexing}]
position = 3, 2
game_map = Map(width=10, height=10)

assert isinstance(
	game_map[position],
	EmptyField
)
\end{lstlisting}

\subsection{Třída hry $Game$}

Třída \ic|Game| je jednou z~nevyýznamnějších třídy v~objektovém návrhu této aplikace. Zajišťuje zpracování klientské herních akce, deleguje herní povely na instance herních polí a kontroluje přiřazování prázdných polí novým botům. Dále si uchovává čas svého vytvoření a čas poslední modifikace, to kvůli administraci. V~případě aktivního konfiguračního parametru \ic|rounded_game| hlídá pořadí tahů botů a v~případně nesrovnalosti vyhazuje vyjímku \ic|BotNotOnTurn|.

Jádro této třídy tvoří slovník akcí - jedná se o~interní instanci datového typu \ic|dict| uloženého v~lokálním kontextu objektu, viz\fullref{lst:game-actions}. Klíče tohoto slovníku jsou hodnoty enumerace\nameref{subsec:action-enum} a jednotlivými hodnotami jsou reference na funkce zpracovávající akce.

\begin{lstlisting}[caption={Slovník akcí ve třídě $Game$},label={lst:game-actions}]
self._actions = {
    Action.STEP: self._action_step,
    Action.TURN_LEFT: self._action_turn_left,
    Action.TURN_RIGHT: self._action_turn_right,
    Action.WAIT: self._action_wait,
    Action.LASER_BEAM: self._action_laser_beam,
}	
\end{lstlisting}
Implementace a popis jednotlivých metod:

\subsubsection{Metoda akce kroku $\_action\_step$}
\vspace{-10pt}
\begin{lstlisting}[caption={Metoda $Game.\_action\_step$},label={lst:game-action-step}]
def _action_step(self, bot_id, bot_field, **kwargs):
    actual_position = self._bots_positions.get(bot_id)

    bot_orientation = bot_field.orientation
    new_position = get_next_position(
        actual_position,
        bot_orientation
    )
    try:
        new_field = self.map[new_position]
    except OutOfMapError:
        raise MovementError('New position is out of map.')

    if isinstance(new_field, TreasureField):
        raise GameFinished()
    elif isinstance(new_field, BotField):
        raise MovementError('Cannot step on another bot.')
    elif isinstance(new_field, BlockField):
        raise MovementError('Cannot step on block.')

    if self._configuration.battery_game and isinstance(bot_field, LaserBatteryBotField):
        bot_field.drain(bot_field.step_battery_cost)

    self._map[new_position], self._map[actual_position] = bot_field, new_field

    self._bots_positions[bot_id] = new_position
\end{lstlisting}

Na řádcích $2-4$ je nejprve získána aktuální pozice bota a následně jeho orientace. Volání na řádcích $5-8$ zajišťuje zjištění následující pozice ve směru natočení bota vzhledem k~jeho akutální pozici. Následuje test na existenci pozice pomocí bloku \ic|try:| $...$ \ic|except OutOfMapError:|. Podmínka na řádku $14$ testuje, zda cílové pole není instancí \ic|TreasureField|, čehož skutečnost by způsobila vyvolání výjimky \ic|GameFinished| a ukončení hry. Následně se na řádcích $16-19$ ověřuje, zdy je možné na cílové bloky přesunout aktuálního bota.

Na řádku $20$ lze v~podstatě prohlásit akci za \emph{proveditelnou}. Na řádcích $21-21$ je následně pochycena hra s~aktivním konfiguračním parametrem $battery_game$ a poté je již poveden pohyb bota pomocí vyměnění zdrojového a cílového pole v~mapě. Před ukončím této metody je ještě aktualizována pozice bota.

\subsubsection[Metody akcí změny orientace]{Metody akcí změny orientace $\_action\_turn\_left$ a $\_action\_turn\_right$}
\vspace{-10pt}
\begin{lstlisting}[caption={Metoda $Game.\_action\_turn\_left$},label={lst:game-action-turn-left}]
def _action_turn_left(self, bot_field, **kwargs):
    assert isinstance(bot_field, BotField)
    bot_field.rotate(Action.TURN_LEFT)
\end{lstlisting}

Dvojice metod \ic|Game._action_turn_left| a \ic|Game._action_turn_right| je si velmi podobná, jediný rozdíl tkví v~hodnotě předané instanci bota - jedná se buď o~\ic|Action.TURN_LEFT| nebo \ic|Action.TURN_RIGHT|. Implementačně se pouzde předeleguje požadavek na instanci \ic|BotField| předanou v~parametru.

\subsubsection{Metoda akce čekání $\_action\_wait$}
\vspace{-10pt}
\begin{lstlisting}[caption={Metoda $Game.\_action\_wait$},label={lst:game-action-wait}]
def _action_wait(self, bot_field, **kwargs):
    if not self._configuration.battery_game:
        raise ActionError('This game is not a battery game.')

    assert isinstance(bot_field, LaserBatteryBotField)
    bot_field.charge(bot_field.battery_charge)
\end{lstlisting}

Ihned na $2-3$ řádku je otestován parametr konfigurace \ic|battery_game|, protože bez něj nenní tato akce povolena. Pokud je vše v~pořádku, je na instanci bota v~mapě předelegován na řádku $6$ příkaz na čekání a nabíjení.

\subsubsection{Metoda akce laseru $\_action\_laser\_beam$}
\label{subsubsec:action-laser-beam}
\vspace{-10pt}
\begin{lstlisting}[caption={Metoda $Game.\_action\_laser\_beam$},label={lst:game-action-laser-beam}]
def _action_laser_beam(self, bot_field, bot_position, **kwargs):
    if not self._configuration.laser_game:
        raise ActionError('This game is not a laser game.')

    assert isinstance(bot_field, LaserBatteryBotField)
    try:
        bot_field.drain(bot_field.laser_battery_cost)
    except CriticalBatteryLevel:
        raise

    for position in get_positions_in_row(self.map, bot_position, bot_field.orientation):
        field = self.map[position]
        if isinstance(field, self._configuration.default_empty_map_field):
            continue
        if isinstance(field, LaserBatteryBotField):
            field.drain(bot_field.laser_damage)
            break
        if isinstance(field, BlockField):
            self.map[position] = self._configuration.default_empty_map_field()
            break
\end{lstlisting}

Metoda \ic|Game._action_laser_beam| je z~metod zpracovávajících herní akce nejsložitější. Prvně je zkontrolován aktivní konfigurační parametr \ic|laser_game| a následně je na řádku $7$ proveden příkaz na vybítí baterie bota (s~možným vyvoláním výjimky \ic|CriticalBatteryLevel| ze třídy \ic|LaserBatteryBotField|).

\begin{figure}
    {\itshape
    Zápis klíčového slova \ic|raise| v~bloku \ic|except| bez parametru je pouze \uv{syntactic sugar}\footnotemark. Následující dva zápisy jsou si návzájem ekvivalentní:
    }
    \begin{minipage}{.475\textwidth}
    \vspace{10pt}
    \begin{lstlisting}
try:
    pass
except Exception as e:
    raise e
    \end{lstlisting}
    \end{minipage}
    \hfill
    \begin{minipage}{.475\textwidth}
    \vspace{10pt}
    \begin{lstlisting}
try:
    pass
except:
    raise
    \end{lstlisting}
    \end{minipage}
    \vspace{-15pt}
\end{figure}
\footnotetext{Syntactic sugar (syntaktický cukr) je určitá forma zjednodušení syntaxe jazyka určená ke zvýšení čitelnosti či zkrácení zápis zdrojového kódu - například $x += 1$ namísto $x = x + 1$.}

Na řádku $11$ poté začíná samotný proces laserového paprsku. Pomocí funkce \ic|get_positions_in_row| jsou získány všechny pozice směrem předaným v~parametru \ic|orientation| ze startovní pozice předané v~parametru \ic|position|. Těmito pozicemi se následně iteruje a získává se instance herního pole do promněnné \ic|field|. Podmínky na řádcích $13-20$ odlišují tyto 3 situace:
\begin{description}
    \item {\bfseries Pole je instancí výchozího prázdného pole v~mapě} \\
        Tento případ nezpůsobuje žádnou změnu - laser prochází dále.

    \item {\bfseries Pole je instancí třídy \ic|LaserBatteryBotField|} \\
        Dochází k~poškození nepřátelského bota - baterie je tomuto botu vybita o~hodnotu poškození útočícího bota a celý cyklus \emph{iterace je zastaven}.

    \item {\bfseries Pole je instancí třídy \ic|BlockField|} \\
        Dochází ke zničení pole bloku, který je nahrazen výchozím prázdným polem a celý cyklus \emph{iterace je zastaven}.
\end{description}

\subsection{Třídy reprezentující herní bloky}

Na tomto místě je nutno poznamenat, že potomci třídy \ic|Field|, jenž jsou umisťováni do mapy, nejsou zodpovědni za své umístění, tzv. neuchovávají žádné informace o~své pozici, ale pouze informace o~svém stavu, jako jsou například orientace, jméno nebo stav baterie. 

\subsubsection{Společný abstraktní předek $Field$}

\begin{sloppypar}
	Pro všechny instance polí existuje společný abstraktní předek - třída \ic|Field|, ze které dědí všechna možná pole umístitelná do mapy. Samotná třída je potomkem třídy \ic|Exportable|, znamená to tedy, že každý z~potomků této třídy musí implementovat metodu \ic|export|, což zajištuje možnost reprezentace herní struktury i mimo aplikaci.
\end{sloppypar}

\subsubsection{Prázdné pole $EmptyField$}

Jedná se o~prázdnou variantu herního pole, je zodpovědná pouze za svůj korektní export.

\subsubsection{Pole bloku $BlockField$}

Třída \ic|BlockField| je ve své jednoduchosti velmi podobná prázdnému poli. Jediný rozdíl skrývá rozdílný přístup třídy \ic|Game| při herních akcích - na pole s~objektem této třídy není možné vstoupit botem, avšak při zapnutém parametru pro laser hru je možno tento blok zničit. Po zničení je instance pole s~blokem nahrazena instancí prázdného pole - viz detail implementace v~sekci\fullref{subsubsec:action-laser-beam}.

\subsubsection{Pole pro poklad $TreasureField$}

Ve své podstatě se jedná o~pole jako každé jiné v~mapě, avšak má jednu spicální vlastnost. Při vstoupení na něj botem je vyvolána výjimka \ic|GameFinished| (o~struktuře a systému výjimek v~sekci\fullref{subsec:custom-exceptions}) a hra končí. 

\subsubsection{Pole pro boty $BotField$ a $LaserBatteryBotField$}

Třída \ic|BotField| je základní třídou pro instance botů ve hře. Je zodpovědná za udržení své orientace v~vlastnosti \ic|orientation|, jenž je v~konstruktoru omezena na instance třídy \ic|Orientation|. Setter\footnote{Metoda nastavující neveřejný atribut, jenž může zároveň vytvářet sestrikce na hodnotu tohoto atributu.} pro tuto vlastnost není naimplementován, protože existuje metoda \ic|BotField.rotate|, jenž bota na základě předané akce otáčí na požadovaný směr. Dále je tato třída zopovědná za udržení jména pro bota - což je záležitost čistě informativní. Toto jméno je uloženo ve vlastnosti \ic|name| a je možné jej nastavit buďto konstruktorem anebo setterem.

\begin{sloppypar}
    Potomek třídy pole bota \ic|LaserBatteryBotField| je lehce složitější. Kromě orientace a jména je také zodpovědný za správu stavu baterie a ovládání laseru. Konstruktor je tedy rozšířen o~parametr \ic|battery_level| určující výchozí hodnotu nabití baterie, \ic|laser_battery_cost| určující \uv{cenu} vystřelení laseru, \ic|laser_damage| určující hodnotu vybití nepřátelské baterie při zásahu, \ic|step_battery_cost| stanovující \uv{cenu} pohybu bota a také \ic|battery_charge|, o~jehož hodnotu je nabita baterie při nabíjení. Setter vlastnosti \ic|actual_battery_level| je zodpovědný za hlídání stavu baterie, v~případě pokus o~její snížení pod hodnotu $0$ je vyvolána výjimka \ic|CriticalBatteryLevel| (více o~ní v~tabulce\fullref{table:custom-exceptions}).
\end{sloppypar}

\subsection{Systém a struktura vlastních výjimek}
\label{subsec:custom-exceptions}

Proces zpracování herní akce je záležitost, při které může dojít až k~několika desítkám nevalidních nebo hru ukončujících stavů, jako např. bot mimo herní mapu, pohyb bota, který není na řadě, požadavek na neznámého bota, požadavek na neznámou mapu, nesprávná herní akce pro kontext hry anebo hra s~již zaplněnými poli pro boty. Proto je v~aplikaci navržen systém výjimek, které jsou vyhazovány v~momentech, kdy je již proces zpracování akce zahájen a nelze zastavit; jednotlivé výjimky jsou umístěny v~tabulce\fullref{table:custom-exceptions}.

{
	\newcolumntype{B}{>{\collectcell\icmacro}r<{\endcollectcell}}
	\renewcommand{\arraystretch}{1.2}
	\centering
	
	\begin{longtable}{ B || p{.6\textwidth} }
        \caption{Seznam vlastních výjimek a jejich popis}\label{table:custom-exceptions}\\

		OutOfMapError & Tuto výjimku používá třída \ic|Map| k~indikaci přístupu mimo rozsah mapy. Kromě utility \ic|get_positions_in_row| ji používá především třída \ic|Game| při procesu zpracování herní akce. \\

		BotNotOnTurn & Výjimka tohoto typu je určena k~indikaci stavu, kdy je vznesen požadavek na tah bota, který ve hře se zapnutým parametrem \ic|rounded_game| není na řadě. \\

		NoFreeBots & Tento typ vyjímky slouží k~rozpoznání stavu, kdy se řídící třída pokusí přidat bota do hry, v~níž již není volné pole pro bota. \\

		ConfigurationError & \ic|ConfigurationError| se používá pro signalizaci stavu nevalidní konfigurace. Ve dvou případech je užívána třídou \ic|MapFactory|, jenž její pomocí rozlišuje stavy při nevalidním požadavku na vytvoření mapy (nemožnost umístit příliš moc herních bloků do příliš malé mapy) a nelogickou konfiguraci parametrů \ic|battery_game = True| a \ic|laser_game = False|. Pomocí této výjimky se také kontroluje validita configuračního formuláře - více o~tomto formuláři v~sekci\fullref{subsec:administration}. \\

		GameFinished & Tato výjimka ukončuje hru - je vyvolána třídou \ic|Game| při vstoupení bota na pole třídy \ic|TreasureField|. Tuto výjimku očekává instance třídy \ic|GameController| v~případě jejího zachycení zaznamenává tuto hru za ukončenou a bota, jenž provedl tento krok za výherce hry. \\

		CriticalBatteryLevel & Výjimku, jenž je potomkem výjimky \ic|MovementError|, vyvolává třída pole \ic|LaserBatteryBotField| v~případě požadavku na vybití baterie pod nulovou úroveň. Vzhledem k~jejímu rodiči je tedy tato výjimka odchytávána jako nemožnost provést herní akci resp. nemožnost provést krok ve hře. \\

	\end{longtable}
}

\subsection{Testy}
\begin{sloppypar}
    V~této aplikaci jsem se snažil klást důraz na \textbf{co nejvyšší pokrytí} zdrojového kódu \textbf{jednotkovými testy}. Bylo tedy nutné navrhnout testy tak, aby byly schopné pokrýt všechny testované třídy a případy. V~kořenovém adresáři projektu tedy vznikl Python balíček \ic|tests| obsahující identickou adresářovou i souborovou strukturu, jako samotné zdrojové kódy ve složce \ic|pybots| - znamená to, že například třída \ic|pybots.game.GameController| má $TestCase$ \ic|tests.pybots.game.TestGameController|. Kromě modulu \ic|tests| vznikl i skript s~příznačným jménem \ic|test.py| zajišťující spuštění balíčku \ic|unittest|, jenž zajištuje spouštění testů v~Pythonu - více v~sekci\fullref{subsec:python}.
\end{sloppypar}

\subsubsection{Vlastní $TestCase$}

V~testech pro tuto aplikaci používám vlastní poděděnou třídu pro množinu testovacích metod \ic|TestCase|. Je rozšířená o~tyto metody s~těmito hlavičkami:
\begin{description}
	\item \ic|assertIsInMap(self, game_map, field_cls, expected_count=1)| \\
	Tato metoda rozšiřuje testovací aserce v~původní testovací třídě \ic|TestCase| o~možnost testovat herní mapu (instanci třídy \ic|Map|) na určitý počet instací herních polí, resp. instancí třídy předané v~parametru \ic|field_cls|. Ve výsledku je provedena aserce počtu nalezených těchto bloků vůči hodnotě parametru \ic|expected_count| s~výchozí hodnotou \ic|1|.

    \item \ic|setUp(self)| \\
    Framework \nameref{subsec:flask} používá ke generování URL adres funkci \ic|url_for|, jenž pro svou funkci potřebuje aktivní objekt \ic|RequestContext| v~globálním Flask objektu \ic|_request_ctx_stack| třídy \ic|LocalStack|. V~této metodě, jenž je volána při spuštění \ic|TestCase| se tedy tento \ic|RequestContext| nastavuje. 

    \item \ic|tearDown(self)| \\
    Tato metoda je volána při ukončení testů z~jedné třídy \ic|TestCase| a je v~ní z~globálního objektu \ic|_request_ctx_stack| třídy \ic|LocalStack| odebrán kontext, jenž byl nastavený v~metodě \ic|setUp|.

    \item \ic|test_client(self)| \\
    Tato metoda vrací nakonfigurovaný objekt třídy \ic|FlaskClient|, jenž je schopen pokládat požadavky na instanci aplikace frameworku \nameref{subsec:flask} - základní informací pro tento požadavek je URL, možné je přidat parametry jak pro \nameref{subsec:http} metodu $GET$, tak pro metodu $POST$. Metoda \ic|test_client| provádí s~tímto objektem jedinou úpravu - \ic|app_client.testing = True|, jenž zajišťuje probublávání výjimek skrz celý strom volání - nezachycené výjimky jdou poté vidět přímo v~testech.

    \item \ic|set_conf_to_test_client(self, conf, client)| \\
    Pomocí této metody lze konfigurovat běžící aplikaci přímo v~testech - volání této metody způsobí odeslání POST požadavku na URL adresu administrace s~daty předanými v~parametru \ic|conf|. Je přitom použit testovací klient předaný v~parametru \ic|client|. Výhodou této metody je především možnost nakonfigurovat aplikaci před každým testem, je možnost nastavit kterýkoliv z~parametrů uvedených v~tabulce\fullref{table:conf-parameters}. 
\end{description}

\subsection{Administrace}
\label{subsec:administration}

Administrační formulář zajištuje třída \ic|ConfigurationForm| poděděná z~obecné třídy \ic|Form| pocházející z~balíčku \ic|wtforms| zajištující generování, typovost, validaci a vykreslování formulářů do \href{https://cs.wikipedia.org/wiki/HyperText_Markup_Language}{HTML}\footnote{\href{https://cs.wikipedia.org/wiki/HyperText_Markup_Language}{HyperText Markup Language} je značkovací jazyk určený pro tvorbu webových stránek vycházející z~jazyka XML.}.

\fullref{lst:conf-form-fields} na řádcích $2-7$ znázorňuje statický zápis jednotlivých vstupů formuláře - jako třída je použita \ic|IntegerField|, reps. \ic|BooleanField|, s první parametrem textovým popisem pole a dalšími parametry, jako jsou \ic|validators| a \ic|filters|.

\begin{lstlisting}[caption={Ukázka deklarace polí konfiguračního formuláře},label={lst:conf-form-fields}]
class ConfigurationForm(Form):
    map_width = IntegerField('map width',
        validators=(DataRequired(), NumberRange(5, 100)),
        filters=(int, ))
    # ...
    laser_game = BooleanField('laser game',
        default=False, validators=(),
        filters=(bool, ))
\end{lstlisting}

Pojmenovaný parametr \ic|validators| příjmá iterovatelný objekt (v tomto případě \ic|tuple|) funkcí (nebo objektů implementující interní metodu \ic|__call__|), jenž jsou volány při validaci samotných prvků. V případě, že je z jakéhokoliv volání vyvolána výjimka \ic|ValueError|, je validace zastavena a prohlášena za neúspěšnou. Konfigurační pole \ic|map_width| používá dva tyto validátory - \ic|DataRequired()| zajištující vyplnění pole a \ic|NumberRange(5, 100)| hlídající rozsah zadané hodnoty. S parametrem \ic|filters| se pracuje podobně, avšak se jednotlivé funkce volají jako filtry hodnoty z pole - používá se \ic|int| pro přetypování hodnoty na stejnojmenný datový typ.

\begin{lstlisting}[caption={Implementace vlastních metod v $ConfigurationForm$},label={lst:conf-form-methods}]
class ConfigurationForm(Form):
    @property
    def as_configuration(self):
        if self.data.get('maze_game'):
            return CustomMazeConfiguration(**self.data)
        return CustomConfiguration(**self.data)

    def validate(self):
        if not super().validate():
            return False
        try:
            MapFactory.create(self.as_configuration)
        except ConfigurationError as e:
            self.bots.errors.append(str(e))
            return False
        return True
\end{lstlisting}
